\documentclass[PhD]{iitmdiss}
%\documentclass[MS]{iitmdiss}
%\documentclass[MTech]{iitmdiss}
%\documentclass[BTech]{iitmdiss}
\usepackage{times}
 \usepackage{t1enc}


\usepackage{epstopdf}
\usepackage[hypertex]{hyperref} % hyperlinks for references.
\usepackage{amsmath} % easier math formulae, align, subequations \ldots

\begin{document}


%%%%%%%%%%%%%%%%%%%%%%%%%%%%%%%%%%%%%%%%%%%%%%%%%%%%%%%%%%%%%%%%%%%%%%
% Title page

\title{Advanced numerical algorithm for solving initial value problem of ordinary differential euations}

\author{Deepshikha Mishra}

\date{August 2023}
\department{MATHEMATICS AND COMPUTING}

%\nocite{*}
\maketitle
% Certificate
\certificate

\vspace*{0.5in}

\noindent This is to certify that the thesis titled {\bf ADVANCED NUMERICAL ALGORITHM FOR SOLVING INITIAL VALUE PROBLEM OF ORDINARY DIFFERENTIAL EQUATION SUBMITTED TO IIT(ISM) DHANBAD}, submitted by {\bf Deepshikha Mishra}, 
  to the Indian Institute of Technology, Dhanbad, for
the award of the degree of {\bf Doctor of Philosophy}, is a bona fide
record of the research work done by him under our supervision.  The
contents of this thesis, in full or in parts, have not been submitted
to any other Institute or University for the award of any degree or
diploma.

\vspace*{1.5in}

\begin{singlespacing}
\hspace*{-0.25in}
\parbox{2.5in}{
\noindent {\bf Prof. Ashish Bhatt} \\
\noindent Research Guide \\ 
\noindent Assistent Professor \\
\noindent Dept. of Mathematics and computing\\
\noindent IIT-DHANBAD, 826 004 \\
} 
\hspace*{1.0in} 
%\parbox{2.5in}{
%\noindent {\bf Prof.~S.~C.~Rajan} \\
%\noindent Research Guide \\ 
%\noindent Assistant Professor \\
%\noindent Dept.  of  Aerospace Engineering\\
%\noindent IIT-Madras, 600 036 \\
%}  
\end{singlespacing}
\vspace*{0.25in}
\noindent Place: Dhanbad



%%%%%%%%%%%%%%%%%%%%%%%%%%%%%%%%%%%%%%%%%%%%%%%%%%%%%%%%%%%%%%%%%%%%%%
% Acknowledgements
\acknowledgements

Thanks to all those who made \TeX\ and \LaTeX\ what it is today.

%%%%%%%%%%%%%%%%%%%%%%%%%%%%%%%%%%%%%%%%%%%%%%%%%%%%%%%%%%%%%%%%%%%%%%
% Abstract

\abstract

\noindent KEYWORDS: \hspace*{0.5em} \parbox[t]{4.4in}{\LaTeX ; Thesis;
  Style files; Format.}

\vspace*{24pt}

\noindent A \LaTeX\ class along with a simple template thesis are
provided here.  These can be used to easily write a thesis suitable
for submission at IIT-Madras.  The class provides options to format
PhD, MS, M.Tech.\ and B.Tech.\ thesis.  It also allows one to write a
synopsis using the same class file.  Also provided is a BIB\TeX\ style
file that formats all bibliography entries as per the IITM format.

The formatting is as (as far as the author is aware) per the current
institute guidelines.

\pagebreak

%%%%%%%%%%%%%%%%%%%%%%%%%%%%%%%%%%%%%%%%%%%%%%%%%%%%%%%%%%%%%%%%%
% Table of contents etc.

\begin{singlespace}
\tableofcontents
\thispagestyle{empty}

\listoftables
\addcontentsline{toc}{chapter}{LIST OF TABLES}
\listoffigures
\addcontentsline{toc}{chapter}{LIST OF FIGURES}
\end{singlespace}


%%%%%%%%%%%%%%%%%%%%%%%%%%%%%%%%%%%%%%%%%%%%%%%%%%%%%%%%%%%%%%%%%%%%%%
% Abbreviations
\abbreviations

\noindent 
\begin{tabbing}
xxxxxxxxxxx \= xxxxxxxxxxxxxxxxxxxxxxxxxxxxxxxxxxxxxxxxxxxxxxxx \kill
\textbf{IITM}   \> Indian Institute of Technology, Madras \\
\textbf{RTFM} \> Read the Fine Manual \\
\end{tabbing}

\pagebreak

%%%%%%%%%%%%%%%%%%%%%%%%%%%%%%%%%%%%%%%%%%%%%%%%%%%%%%%%%%%%%%%%%%%%%%
% Notation

\chapter*{\centerline{NOTATION}}
\addcontentsline{toc}{chapter}{NOTATION}

\begin{singlespace}
\begin{tabbing}
xxxxxxxxxxx \= xxxxxxxxxxxxxxxxxxxxxxxxxxxxxxxxxxxxxxxxxxxxxxxx \kill
\textbf{$r$}  \> Radius, $m$ \\
\textbf{$\alpha$}  \> Angle of thesis in degrees \\
\textbf{$\beta$}   \> Flight path in degrees \\
\end{tabbing}
\end{singlespace}

\pagebreak
\clearpage

% The main text will follow from this point so set the page numbering
% to arabic from here on.
\pagenumbering{arabic}


%%%%%%%%%%%%%%%%%%%%%%%%%%%%%%%%%%%%%%%%%%%%%%%%%%
% Introduction.

\chapter{INTRODUCTION}
\label{chap:intro}

This document provides a simple template of how the provided
\verb+iitmdiss.cls+ \LaTeX\ class is to be used.  Also provided are
several useful tips to do various things that might be of use when you
write your thesis.

Before reading any further please note that you are strongly advised
against changing any of the formatting options used in the class
provided in this directory, unless you are absolutely sure that it
does not violate the IITM formatting guidelines.  \emph{Please do not
  change the margins or the spacing.}  If you do change the formatting
you are on your own (don't blame me if you need to reprint your entire
thesis).  In the case that you do change the formatting despite these
warnings, the least I ask is that you do not redistribute your style
files to your friends (or enemies).

It is also a good idea to take a quick look at the formatting
guidelines.  Your office or advisor should have a copy.  If they
don't, pester them, they really should have the formatting guidelines
readily available somewhere.

To compile your sources run the following from the command line:
\begin{verbatim}
% latex thesis.tex
% bibtex thesis
% latex thesis.tex
% latex thesis.tex
\end{verbatim}
Modify this suitably for your sources.

To generate PDF's with the links from the \verb+hyperref+ package use
the following command:
\begin{verbatim}
% dvipdfm -o thesis.pdf thesis.dvi
\end{verbatim}

\section{Package Options}

Use this thesis as a basic template to format your thesis.  The
\verb+iitmdiss+ class can be used by simply using something like this:
\begin{verbatim}
\documentclass[PhD]{iitmdiss}  
\end{verbatim}

To change the title page for different degrees just change the option
from \verb+PhD+ to one of \verb+MS+, \verb+MTech+ or \verb+BTech+.
The dual degree pages are not supported yet but should be quite easy
to add.  The title page formatting really depends on how large or
small your thesis title is.  Consequently it might require some hand
tuning.  Edit your version of \verb+iitmdiss.cls+ suitably to do this.
I recommend that this be done once your title is final.

To write a synopsis simply use the \verb+synopsis.tex+ file as a
simple template.  The synopsis option turns this on and can be used as
shown below.
\begin{verbatim}
\documentclass[PhD,synopsis]{iitmdiss}                                
\end{verbatim}

Once again the title page may require some small amount of fine
tuning.  This is again easily done by editing the class file.

This sample file uses the \verb+hyperref+ package that makes all
labels and references clickable in both the generated DVI and PDF
files.  These are very useful when reading the document online and do
not affect the output when the files are printed.


\section{Example Figures and tables}

Fig.~\ref{fig:iitm} shows a simple figure for illustration along with
a long caption.  The formatting of the caption text is automatically
single spaced and indented.  Table~\ref{tab:sample} shows a sample
table with the caption placed correctly.  The caption for this should
always be placed before the table as shown in the example.


\begin{figure}[htpb]
  \begin{center}
    \resizebox{50mm}{!} {\includegraphics *{iitm.eps}}
    \resizebox{50mm}{!} {\includegraphics *{iitm.eps}}
    \caption {Two IITM logos in a row.  This is also an
      illustration of a very long figure caption that wraps around two
      two lines.  Notice that the caption is single-spaced.}
  \label{fig:iitm}
  \end{center}
\end{figure}

\begin{table}[htbp]
  \caption{A sample table with a table caption placed
    appropriately. This caption is also very long and is
    single-spaced.  Also notice how the text is aligned.}
  \begin{center}
  \begin{tabular}[c]{|c|r|} \hline
    $x$ & $x^2$ \\ \hline
    1  &  1   \\
    2  &  4  \\
    3  &  9  \\
    4  &  16  \\
    5  &  25  \\
    6  &  36  \\
    7  &  49  \\
    8  &  64  \\ \hline
  \end{tabular}
  \label{tab:sample}
  \end{center}
\end{table}

\section{Bibliography with BIB\TeX}

I strongly recommend that you use BIB\TeX\ to automatically generate
your bibliography.  It makes managing your references much easier.  It
is an excellent way to organize your references and reuse them.  You
can use one set of entries for your references and cite them in your
thesis, papers and reports.  If you haven't used it anytime before
please invest some time learning how to use it.  

I've included a simple example BIB\TeX\ file along in this directory
called \verb+refs.bib+.  The \verb+iitmdiss.cls+ class package which
is used in this thesis and for the synopsis uses the \verb+natbib+
package to format the references along with a customized bibliography
style provided as the \verb+iitm.bst+ file in the directory containing
\verb+thesis.tex+.  Documentation for the \verb+natbib+ package should
be available in your distribution of \LaTeX.  Basically, to cite the
author along with the author name and year use \verb+\cite{key}+ where
\verb+key+ is the citation key for your bibliography entry.  You can
also use \verb+\citet{key}+ to get the same effect.  To make the
citation without the author name in the main text but inside the
parenthesis use \verb+\citep{key}+.  The following paragraph shows how
citations can be used in text effectively.

More information on BIB\TeX\ is available in the book by
\cite{lamport:86}.  There are many
references~\citep{lamport:86,prabhu:xx} that explain how to use
BIB\TeX.  Read the \verb+natbib+ package documentation for more
details on how to cite things differently.

Here are other references for example.  \citet{viz:mayavi} presents a
Python based visualization system called MayaVi in a conference paper.
\citet{pan:pr:flat-fst} illustrates a journal article with multiple
authors.  Python~\citep{py:python} is a programming language and is
cited here to show how to cite something that is best identified with
a URL.

\section{Other useful \LaTeX\ packages}

The following packages might be useful when writing your thesis.

\begin{itemize}  
\item It is very useful to include line numbers in your document.
  That way, it is very easy for people to suggest corrections to your
  text.  I recommend the use of the \texttt{lineno} package for this
  purpose.  This is not a standard package but can be obtained on the
  internet.  The directory containing this file should contain a
  lineno directory that includes the package along with documentation
  for it.

\item The \texttt{listings} package should be available with your
  distribution of \LaTeX.  This package is very useful when one needs
  to list source code or pseudo-code.

\item For special figure captions the \texttt{ccaption} package may be
  useful.  This is specially useful if one has a figure that spans
  more than two pages and you need to use the same figure number.

\item The notation page can be entered manually or automatically
  generated using the \texttt{nomencl} package.

\end{itemize}

More details on how to use these specific packages are available along
with the documentation of the respective packages.

%%%%%%%%%%%%%%%%%%%%%%%%%%%%%%%%%%%%%%%%%%%%%%%%%%%%%%%%%%%%
% Appendices.

\appendix
\chapter{TIMELINE FOR PHD}

Having my long term goals in my mind, my aim is to work towards a good in
mathematics. Next five years, I am looking forward to focus on research areas that
can extend the availability and usefulness of technology and its advancement to our
society directly. I believe that this will certainly enhance their quality of livings.
My first research problem is based on geometric numerical integration of ordinary
differential equation.
I divide my time period in ten sections(semester wise).

    {\textbf{FIRST SEMESTER}:{ In this duration I have to complete the following tasks.}\\
    \begin{itemize}
 
    
    \item{Complete five subjects (analysis, numerical method, differential equation, re
search methodology, research and technical communication) of the coursework.}\\
\item{Learn some basic tools and programming language as: MATLAB, PYTHON,
GIT, COMMAND LINE, LATEX.}\\
\item{Try to improve my writing skill, communication skill, and also my English
language.}\\
\item{Complete at least two chapter of the book geometric numerical integration with
short notes.}\\
\item{With the help of litrature review, try to find some new ideas.}\\
\end{itemize}
{\textbf{SECOND SEMESTER}:{In second semester, I will complete followings.}\\
\begin{itemize}
    

    \item{Complete the remaining coursework.\\
 \item{ Complete the book geometric numerical integration with short notes.}\\
 \item{Getting my research problem and a rough idea about writing research paper.}\\
 \item{At last of this semester (June-2020) try to start writing research paper.}\\
 
\end{itemize}
\textbf{THIRD SEMESTER} : In third semester,I have to do a paper at any cost.I want to do my first paper on
ACTA NUMERICA JOURNAL. Throughout this time, my entire focus will be on first paper
and I will do it diligently.\\
\par
\textbf{FOURTH SEMESTER} :After the completion of first paper, my focus will be on presenting the paper on some international conference. Meanwhile I will also think about my second research problem. In this time span I will also try to attend at least a workshop to explore some new ideas.}\\

\chapter{CHAPTER 1\\ EXAMPLES AND NUMERICAL EXPERIMENTS}
Any mathematical model associated with real world situation can be categorized in two forms.Model is either in form of system of differential equation (flows of the system) or system of difference equation(iterated maps).In this chapter we will make a better understanding of system of ordinary differential equation, more specifically system of first order ordinary differential equation. The mathematical model modeled by us often occurs in nonlinear form.Evidently, these problems can not be done analytically.So In this segment we will analyze some numerical experiments and will try to do something new.
There are some fundamental techniques to solve system of first order ordinary differential equation.\\
\begin{itemize}
    \item {Explicit Euler Method}\\
    \item {Implicit Euler Method}\\
    \item {Implicit Midpoint Rule}\\
    \item {Symplectic Euler Methods}\\
    \item {The Stormer Verlet Scheme}\\
\end{itemize}
All these methods have their origin in Explicit Euler Method. Some elementary models  defined in this chapter by which we can understand these methods well.
\begin{itemize}
    \item {\textbf{The Lotka Volterra Model} : This model describes the dynamics of biological system in which two species interact as prey and predator. Solution of lotka volterra model is periodic because of existence of invariant and the most important thing is that the explicit and implicit Euler methods show wrong qualitative behaviour while the symplectic euler method preserves the exact solution.}\\
    \item{\textbf{Hamiltonian System} : Solution of hamiltonian system is also periodic because in this case hamiltonian is an invariant. For hamiltonian system, Stormer Verlet method shows exact qualtitatve behaviour while other methods do not.}\\
    
\end{itemize}
\chapter{CHAPTER 2\\NUMERICAL INTEGRATORS}
This chapter covers some advanced numerical techniques than previous ones and these are:\\
\begin{enumerate}
    \item{Runge kutta method}
    \item{Collocation method}
    \item{Discontinuous collocation method}
    \item{partitioned runge kutta method}
    \item{Nystrom method}
    \item{Composition method}
    \item{Splitting method}
\end{enumerate}
Every methods have their own importance so now we will elaborate all the methods.
\begin{enumerate}
    \item {\textbf{Runge kutta method:} The main feature of the runge kutta method is averaging the four slopes at three points of the interval [a,b]. First slope is taken at a, second and third at midpoint of the interval i.e. $(a+b)/2$  and the last fourth is taken at b.}
    \item{\textbf{Collocation method:} Among all numerical techniques for numerical integration, Collocation method is quite specific. Since every method used to get a discrete set of approximation whether Collocation method gives a continuous set of approximation. So let us assume that we have to find area under the curv f(x) (i.e. $\int_a^b{f(x) dx}$). In Collocation method, we are concerned about finding f(x) on [a,b]. First we choose three points : a, (a+b)/2 and b and the connecting these three points through a second degree polynomial will give collocation polynomial. Collocation polynomial is nothing but approximated f(x). This method is known as Collocation method.\\}
    \item{\textbf{Discontinuous collocation method: } The discontinuous collocation methods are equivalent to implicit Runge kutta method.}
    \item{\textbf{Partitioned runge kutta method:} Partitioned runge kutta method is better approach to solving second order differential equation using runge kutta scheme. Here the partitioned means that separate treatment for two or more dependent variable with respect to one independent variable. Symplectic euler and Stormer verlet are also partitioned type method.}
    \item{\textbf{Nystrom method:} Modifying partitioned runge kutta method in a certain way (as changing the coefficient) Nystrom method can be obtained.}
    
    
    
    
    
    
    
    
    
    
    
\end{enumerate}




%%%%%%%%%%%%%%%%%%%%%%%%%%%%%%%%%%%%%%%%%%%%%%%%%%%%%%%%%%%%
% Bibliography.

\begin{singlespace}
  \bibliography{refs}
\end{singlespace}


%%%%%%%%%%%%%%%%%%%%%%%%%%%%%%%%%%%%%%%%%%%%%%%%%%%%%%%%%%%%
% List of papers

\listofpapers

\begin{enumerate}  
\item Authors....  \newblock
 Title...
  \newblock {\em Journal}, Volume,
  Page, (year).
\end{enumerate}  

\end{document}
